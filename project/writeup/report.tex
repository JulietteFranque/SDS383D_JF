\documentclass[paper=a4, fontsize=11pt]{scrartcl}
\usepackage[T1]{fontenc}
\usepackage{fourier}
\usepackage[numbers]{natbib}
\usepackage[english]{babel}	
% English language/hyphenation
\usepackage[protrusion=true,expansion=true]{microtype}	
\usepackage{amsmath,amsfonts,amsthm} % Math packages
\usepackage[pdftex]{graphicx}	
\usepackage{url}
\usepackage[
backend=biber,
style=numeric,
sorting=ynt
]{biblatex}
\addbibresource{refs.bib}
\usepackage{enumitem}

\newlength\tindent
\setlength{\tindent}{\parindent}
\setlength{\parindent}{0pt}
\renewcommand{\indent}{\hspace*{\tindent}}

%%% Custom sectioning
\usepackage{sectsty}
\allsectionsfont{\centering \normalfont\scshape}


%%% Custom headers/footers (fancyhdr package)
\usepackage{fancyhdr}
\pagestyle{fancyplain}
\fancyhead{}											% No page header
\fancyfoot[L]{}											% Empty 
\fancyfoot[C]{}											% Empty
\fancyfoot[R]{\thepage}									% Pagenumbering
\renewcommand{\headrulewidth}{0pt}			% Remove header underlines
\renewcommand{\footrulewidth}{0pt}				% Remove footer underlines
\setlength{\headheight}{13.6pt}


%%% Equation and float numbering
\numberwithin{equation}{section}		% Equationnumbering: section.eq#
\numberwithin{figure}{section}			% Figurenumbering: section.fig#
\numberwithin{table}{section}				% Tablenumbering: section.tab#


%%% Maketitle metadata
\newcommand{\horrule}[1]{\rule{\linewidth}{#1}} 	% Horizontal rule

\title{
		%\vspace{-1in} 	
		\usefont{OT1}{bch}{b}{n}
		\normalfont \normalsize \textsc{The University of Texas at Austin} \\ [25pt]
		\horrule{0.5pt} \\[0.4cm]
		\huge Statistical Modeling II Project \\
		\horrule{2pt} \\[0.5cm]
}
\author{
		\normalfont 								\normalsize
        Juliette Franqueville\\[-3pt]		\normalsize
        \today
}
\date{}


%%% Begin document
\begin{document}
\maketitle
\newpage
\section{Introduction}

The COVID-19 pandemic resulted in stay-at-home orders across the world. As a result, people spent more time at home and in outdoor areas and less time commuting to work or restaurants. This report explores the relationship between the changes in mobility patterns of the general population and the number of incidents reported by fire departments in the United States in 2020. Several studies have already explored the effect of COVID-19 on fire incidents. For example, Suzuki and Manzello \cite{fire_paper} analyzed the effect of stay-at-home orders on cooking fires in major cities. Koester and Greatbatch \cite{koester2020comparing} investigated the impact of the onset of the pandemic on fire and search and rescue incident frequency. This report utilizes NFORS \cite{nfors} analytics data provided by the International Public Safety Data Institute and 2020 Google Mobility data \cite{google}. The names of the fire departments were hidden for anonymity. 

\section{Data Formatting}
The raw datasets were formatted to obtain weekly averages for number of incident percentage change from baseline for 2020 for each fire department in NFORS and 2020 weekly average Google mobility data corresponding to the county of each NFORS department.\\


The Google data columns of interest were date,  FIPS code (which corresponds to unique county) and percentage change from baseline for workplace, residential, retail, parks, and grocery/pharmacy mobility. First, the Google mobility data were grouped by FIPS codes.  The data with FIPS codes corresponding to those of the counties of the fire departments in the NFORS data were kept.\\



The Google mobility data is reported as a percentage change from baseline. The baseline used is the median value for each day of the weeks over the January 3rd - February 6th 2020 period. To get a weekly average, a 7-day rolling average was used and the value for all Mondays was reported as the weekly average. \\

In the NFORS data, the data of interest were date, fire department, and number of incidents reported for each date. For each fire department, gaps in data (where more than two consecutive days had zero incidents, which is very unlikely) were removed. Then, outliers were removed by fitting a negative binomial distribution to the incident counts of each department using moment matching and removing points outside the 95 \% confidence interval.\\


An incident baseline was calculated from the 2019 NFORS data and applied to the 2020 data. The baseline was calculated by taking the median of the number of incidents per day of the week per month for each department. Accounting for month as well as day of the week ensured that the seasonality in fire incidents was accounted for. Then, the percentage change for each day in 2020 was calculated from this baseline. Note that some departments were missing significant amounts of data and did not allow for calculating a baseline for each month and each day of the week in 2019. For missing baseline data points, the corresponding changes from baseline in 2020 could not be calculated.  Some department had no data for 2019, so they were not used in the analysis, As with the Google data, a weekly average for incident percentage change form baseline for each department was calculated. \\

 The resulting data format was an $X_i \in \mathbb{R}^{n_i \times p}$ matrix for each department, where $p$ is the number of mobility types (workplace, retails, etc), $i$ is the department, and $n_i$ is the number of weekly averages for each department. Note that ideally, each department would have $n_i = 52$ because there are 52 weeks in a year, but the Google mobility data only begins in February 2020 and some departments had missing data. Each department also had a $y_i \in \mathbb{R}^{n_i}$ vector of incident percentage change from baseline corresponding to the $X_i$ matrices. Additionally, each department had a $t_i \in \mathbb{R}^{n_i}$ vector containing the day of the year that the observations are for (days all correspond to Mondays because the weekly averages were evaluated on Mondays).


\section{Data Exploration}
The various mobility types were highly correlated. Figure \ref{corr_mat} shows the correlation matrix for the mobility types for all departments. The correlations are as expected; for example workplace mobility is inversely correlated with residential mobility. Parks, however, is not highly correlated with the other mobility types.


\begin{figure}[h]\label{corr_mat}
\centering
\includegraphics[width=.7\textwidth]{corr.png}
\caption{Correlation Matrix}
\end{figure}

To avoid multicolinearity issues in the model, principal component analysis was performed and the first two components were kept. The components were calculated by first subtracting the mean of each mobility type from the corresponding mobility values. Then, the covariance matrix for the centered data was calculated. Then, the eigenvalues (or the explained variance corresponding to each component) and the eigenvectors (components) of the covariance matrix were calculated. The two components corresponding to the two highest eigenvalues were kept. The first two components explained 99\% of the variance. The first component  pointed in the direction of ``parks'' and the second pointed in the direction of ``retail/recreation'', ``grocery/pharmacy'', ``workplaces'' and away from ``residential''. The first component can easily be interpreted as park mobility, and the second as increasing ``city'' mobility and decreasing residential mobility. The original data were transformed using the chosen components.\\

The next step was to determine whether the incident data could be approximated as being normally distribution, where:
\begin{align*}
    y_i \sim N(X_i \beta_i, \sigma_i^2I)
\end{align*}
Where $i$ is the fire department, $X_i$ is the design matrix including an intercept and transformed mobility data, $y_i$ is the response vector, $\beta_i \in \mathbb{R}^{p}$, and $I$ is the $n_i \times n_i$ identity matrix.  A simple linear regression model was fit to the data for each department. As shown in Figure \ref{normal}, the Q-Q plots showed  that the residuals were approximately normally distributed; therefore no further transformations of the data were necessary.\\

Another concern was that the residuals (as plotted in Figure \ref{normal} - although some departments showed worse correlation than the 9 plotted) may be auto-correlated, in which case assuming that the covariance matrix was diagonal would not have been appropriate.
    
    
\begin{figure}[h]\label{normal}
\centering
\includegraphics[width=.55\textwidth]{normal_approx_1.png}
\caption{Plots showing residuals with fitted normal with same moments as the residuals (left), Q-Q plot (middle), residuals against time (right). Only the first 9 departments are shown.}
\end{figure}


\section{Model}
\section{Results}





\printbibliography

\end{document}